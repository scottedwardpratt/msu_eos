\documentclass[12pt]{article}
\usepackage{subfiles}
\usepackage{graphicx}
\usepackage{relsize}
\usepackage[
        pdfencoding=auto,%
        pdfauthor={Scott Pratt},%
        pdfstartview=FitV,%
        colorlinks=true,%
        linkcolor=blue,%
        citecolor=blue, %
        urlcolor=blue,
        breaklinks=true]{hyperref}
%\usepackage[anythingbreaks,hyphenbreaks]{breakurl}
\usepackage[anythingbreaks,hyphenbreaks]{xurl}
%\usepackage{pdfsync}
\usepackage{amssymb}
\usepackage{amsmath}
\usepackage{bm}
\numberwithin{equation}{section} 
\numberwithin{figure}{section} 
\usepackage[small,bf]{caption}
%\usepackage{fontspec}
%\usepackage{textcomp}
%\usepackage{color}
\usepackage{fancyhdr}
\setlength{\headheight}{16pt}
%\usepackage[headheight=110pt]{geometry}
\usepackage{bm}

\usepackage[most]{tcolorbox}
\tcbset{
frame code={}
center title,
left=0pt,
right=0pt,
top=0pt,
bottom=0pt,
colback=gray!25,
colframe=white,
width=\dimexpr\textwidth\relax,
enlarge left by=0mm,
boxsep=5pt,
arc=0pt,outer arc=0pt,
}
\newcounter{examplecounter}
\counterwithin{examplecounter}{section}
\setcounter{examplecounter}{0}
\newcommand{\example}[2]{\begin{tcolorbox}[breakable,enhanced]
\refstepcounter{examplecounter}{
\bf Example \arabic{section}.\arabic{examplecounter}:}~~{\bf #1}\\
{#2}
\end{tcolorbox}
}
\textwidth 7.0 in
\hoffset -0.8 in
%\newcommand{\exampleend}{
%\begin{samepage}
%\nopagebreak\noindent\rule{\textwidth}{1pt}
%\end{samepage}
%}


%\usepackage{silence}
%\WarningFilter{hyperref}{Token not allowed in a PDF String}

\newcommand\eqnumber{\addtocounter{equation}{1}\tag{\theequation}}

\newcommand{\solution}[1]{ }



\usepackage{comment}
\parskip 4pt
\parindent 0pt

%\newcommand{\bm}{\boldmath}
\boldmath

%

\begin{document}

\today\\

\centerline{\bf \large Fixed Charge Thermodynamics}
%\centerline{Oleh Savchuk and Scott Pratt}
%\centerline{Michigan State University}

\section{Basing relations on Free-Energy Density}

The Helmholtz free energy density is convenient when writing quantities as functions of the charge density $\rho$ and the temperature $T$. In terms of the canonical partition function,
\begin{align*}\eqnumber
f(\rho,T)&=\frac{1}{V}(E-TS)=-\frac{T}{V}\ln Z.
\end{align*}

The differential free energy density is, after using the identity $dV/V=-d\rho/\rho$,
\begin{align*}\eqnumber
df&=\frac{d\rho}{\rho}f+\frac{1}{V}(-SdT-PdV)\\
&=\frac{d\rho}{\rho}f-sdT+\frac{Pd\rho}{\rho}\\
&=\mu d\rho-sdT.
\end{align*}

The last relation used the identity,
\begin{align*}\eqnumber\label{eq:sept}
s=\beta\epsilon+P/T-\mu\rho/T.
\end{align*}
Reading off the last expression,
\begin{align*}\eqnumber
s&=-\frac{\partial f}{\partial T},~~~\mu=\frac{\partial f}{\partial\rho},
\end{align*}
where $f$ is assumed to be a function of $\rho$ and $T$. Using the definition of $f$,
\begin{align*}\eqnumber
\epsilon&=f+Ts=f+T\frac{\partial f}{\partial T}=-T^2\frac{\partial}{\partial T}\left(\frac{f}{T}\right).
\end{align*}
Using Eq. (\ref{eq:sept}) one can write the pressure,
\begin{align*}\eqnumber
P&=-f+\mu\rho.
\end{align*}
Summarizing, to this point we have the basic thermodynamic quantities in terms of $f(T,\rho)$:
\begin{align*}\eqnumber
\epsilon&=-T^2\frac{\partial}{\partial T}\left(\frac{f}{T}\right),\\
s&=-\frac{\partial f}{\partial T},\\
\mu&=\frac{\partial f}{\partial \rho},\\
P&=-f+\mu\rho.
\end{align*}

\subsection{Susceptibilities and fluctuations}

Next, we need to express the susceptibilities. Consider two matrices,
\begin{align*}\eqnumber
\chi_{ab}&\equiv\frac{\partial\rho_a}{\partial\mu_b},\\
M_{ab}&\equiv\frac{\partial\mu_a}{\partial\rho_b}.
\end{align*}
For the first relation, the derivatives assume the other chemical potentials are held constant (while changing $\mu_b$), whereas the second relation implies that the other densities are held constant. One can rewrite these as,
\begin{eqnarray}
\delta\rho_a&=\chi_{ab}\delta\mu_b,\\
\delta\mu_a&=M_{ab}\delta\rho_b.
\end{eqnarray}
If I multiply the lower expression by the inverse of $M$,
\begin{align*}\eqnumber
\delta\rho_a&=M^{-1}_{ab}\delta\mu_b.
\end{align*}
Thus, one can see that
\begin{align*}
M&=\chi^{-1},~~~\chi=M^{-1}.
\end{align*}
Thus, to get the charge fluctuations in the grand-canonical ensemble (which is our goal) from a formulation based on the $f(T,\rho)$, one can proceed via,
\begin{align*}\eqnumber
\mu_a&=\frac{\partial f}{\partial\rho_a},\\
\chi^{-1}_{ab}&=\frac{\partial^2 f}{\partial\rho_a\partial\rho_b}.
\end{align*}
The charge fluctuation is then found by inverting $\chi^{-1}$.

One can also find the fluctuation of the energy density, at fixed $\mu$, in terms of $f(T,\rho)$.

\begin{align*}\eqnumber
\delta\epsilon&=
\frac{\partial\epsilon}{\partial\rho_a}\delta\rho_a
+\frac{\partial\epsilon}{\partial T}\delta T,\\
\delta\mu_a&=\frac{\partial\mu_a}{\partial\rho_b}\delta\rho_b+\frac{\partial\mu_a}{\partial T}\delta T=0,\\
\delta\rho_a&=-\left(\frac{\partial\mu}{\partial\rho}\right)_{ab}^{-1}\frac{\partial\mu_b}{\partial T}\delta T,\\
&=-\chi_{ab}\frac{\partial\mu_b}{\partial T}\delta T,\\
\delta\epsilon&=\delta T\left\{
-\frac{\partial \epsilon}{\partial\rho_a}\chi_{ab}\frac{\partial\mu_b}{\partial T}+\frac{\partial\epsilon}{\partial T}
\right\}.
\end{align*}
This yields
\begin{align*}\eqnumber
\chi_{EE}&=\frac{\partial\epsilon}{\partial T}
-\frac{\partial\epsilon}{\partial\rho_a}\chi_{ab}\frac{\partial\mu_b}{\partial T}.
\end{align*}

Next, let's calculate $\chi_{EQ}$. We do similar steps above, except set $\delta T=0$.
\begin{align*}\eqnumber
\delta\epsilon&=\frac{\partial\epsilon}{\partial\rho_a}\delta\rho_a,\\
\delta\rho_a&=\chi_{ab}\delta\mu_b,\\
\delta\mu_a&=\chi^{-1}_{ab}\delta\rho_b,\\
\frac{\partial\epsilon}{\partial\mu_a}&=\frac{\partial\epsilon}{\partial\rho_b}\chi_{ba}.
\end{align*}

\subsection{The speed of sound}
When taking partial derivatives, assume functions are of $T$ and $\tilde{\mu}_a\equiv\mu_a/T$.
\begin{align*}\eqnumber
c_s^2&=\left.\frac{\partial P}{\partial\epsilon}\right|_{s/\rho},\\
\delta P&=\frac{\partial P}{\partial T}\delta T+\frac{\partial P}{\partial\tilde{\mu}_a}\delta\tilde{\mu}_a,\\
\delta \epsilon&=\frac{\partial\epsilon}{\partial T}\delta T+\frac{\partial\epsilon}{\partial\tilde{\mu}_a}\delta\tilde{\mu}_a,\\
\delta\rho_a&=\frac{\partial \rho_a}{\partial T}\delta T+\frac{\partial \rho_a}{\partial\tilde{\mu}_b}\delta\tilde{\mu}_b\\
&=\frac{\partial \rho_a}{\partial T}\delta T+\chi_{ab}\delta\tilde{\mu}_b,\\
\delta s&=\frac{\partial s}{\partial T}\delta T+\frac{\partial s}{\partial\tilde{\mu}_a}\delta\tilde{\mu}_a,\\
s^2\delta\left(\frac{\delta\rho_a}{s}\right)&=s\delta\rho_a-\rho_a\delta s=0,\\
0&=s\frac{\partial\rho_a}{\partial T}\delta T+s\chi_{ab}\delta\tilde{\mu}_b
-\rho_b\frac{\partial s}{\partial T}\delta T-\rho_a\frac{\partial s}{\partial\tilde{\mu}_b}\delta\tilde{\mu}_b,\\
\delta\tilde{\mu}_a&=M_a\frac{\delta T}{T},\\
M_a&=T\left(s\chi-\rho\frac{\partial s}{\partial\tilde{\mu}}\right)_{ab}^{-1}
\left(\rho_b\frac{\partial s}{\partial T}-s\frac{\partial\rho_b}{\partial T}\right).
\end{align*}
By substituting for $\delta\tilde{\mu}$ in the expressions for $\delta P$ and $\delta\epsilon$, then taking the ratio $\delta P/\delta\epsilon$, the factors of $\delta T$ drop out and
\begin{align*}\eqnumber
c_s^2&=\frac{1}{T}\frac{P+\epsilon+\rho_aM_a}{\frac{\partial\epsilon}{\partial T}+\frac{\partial\epsilon_a}{\partial\tilde{\mu}_a}M_a}
\end{align*}
Next, substitute for derivative of $s$, so that all derivatives are related to fluctuations.
\begin{align*}
\frac{\partial P}{\partial T}&=\frac{P+\epsilon}{T},\\
\frac{\partial s}{\partial T}&=\frac{\partial}{\partial T}\left(\frac{P+\epsilon-\tilde\mu_a\rho_aT}{T}\right)\\
&=\frac{1}{T}\left(\frac{P+\epsilon}{T}+\frac{\partial\epsilon}{\partial T}-T\tilde{\mu_a}\frac{\partial\rho_a}{\partial T}-\tilde{\mu}_a\rho_a\right)-\frac{s}{T}\\
&=\frac{1}{T}\frac{\partial\epsilon}{\partial T}-\tilde{\mu}_a\frac{\partial\rho_a}{\partial T},\\
\frac{\partial s}{\partial\tilde{\mu}_a}&=\frac{1}{T}\frac{\partial}{\partial\tilde{\mu}_a}(P+\epsilon-\tilde{\mu}_b\rho_bT),\\
&=\frac{1}{T}\left(
\rho_aT+\frac{\partial\epsilon}{\partial\tilde{\mu}_a}-\rho_aT-\tilde{\mu_b}T\frac{\partial\tilde{\mu}_b}{\partial\mu_a}
\right)\\
&=\frac{1}{T}\left(
\frac{\partial\epsilon}{\partial\tilde{\mu}_a}-\tilde{\mu}_bT\frac{\partial\rho_b}{\partial\tilde{\mu}_a}
\right)m\\
M_a&=A^{-1}_{ab}\left(
\rho_b\frac{\partial\epsilon}{\partial T}-Ts\frac{\partial\rho_b}{\partial T}\right),\\
A_{ab}&=s\chi_{ab}-\rho_a\left(
\frac{1}{T}\frac{\partial\epsilon}{\partial\tilde{\mu}_b}-\tilde{\mu}_c\frac{\partial\rho_c}{\partial\tilde{\mu}_b}
\right)\\
&=s\chi_{ab}-\frac{\rho_a}{T}\frac{\partial\epsilon}{\partial\tilde{\mu}_b}+\rho_a\chi_{bc}\tilde{\mu}_c.
\end{align*}
Now, let's summarize the results for the speed of sound using the following definitions for the generalized susceptibilities,
\begin{align*}\eqnumber
\xi&\equiv \frac{\partial\epsilon}{\partial T}=\frac{1}{T^2V}\langle\delta E\delta E\rangle,\\
\zeta_a&\equiv\frac{1}{T}\frac{\partial\epsilon}{\partial\tilde{\mu}_a}=T\frac{\partial\rho_a}{\partial T}
=\frac{1}{TV}\langle\delta E\delta Q_a\rangle,\\
\chi_{ab}&\equiv\frac{\partial\rho_a}{\partial\tilde{\mu}_b}=\frac{1}{V}\langle\delta Q_a\delta Q_b\rangle.
\end{align*}
This gives
\begin{align*}\eqnumber
c_s^2&=\frac{(P+\epsilon)/T+\rho_a M_a}{\xi+\zeta_aM_a},\\
M_a&=A^{-1}_ab(\xi\rho_b-s\zeta_b),\\
A_{ab}&=s\chi_{ab}-\rho_a\zeta_b.
\end{align*}


\end{document}